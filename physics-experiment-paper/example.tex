\documentclass{testreport_class}
%%注意:必须使用xelatex进行编译,否则会编译失败导致无pdf文件生成!
%%注意:必须使用xelatex进行编译,否则会编译失败导致无pdf文件生成!
%%注意:必须使用xelatex进行编译,否则会编译失败导致无pdf文件生成!

\title{这是标题,例如:利用平板电容器测量真空介电常数}
\author{这是作者,物理学系}
\date{}             %本条日期防止\@date命令自动填充
%\date{2024/9/5}    %可能不必加日期

%%%%%%%%%%%%%%%%%%%%%%%引言区结束%%%%%%%%%%%%%%%%%%%%%%%%%%
\begin{document}
	% 页面风格
	\pagestyle{bianyi}

	\maketitle

    \begin{abstruct}
      这里是摘要。
    \end{abstruct}
    \section{\label{Sec.1}引言}
      本模板可以插入article文档类中的各类环境,以下进行一小部分测试,各位可以自行调试。

      模板的类文档文件是testreport\_class.cls,如遇到必须调整的或冲突的,可以自行删改。以下将在第\ref{Sec.2}节中详细介绍模板使用方法以及部分注意事项。
    
      \section{\label{Sec.2}实验原理}

      \subsection{模板的使用和注意事项}

        文件夹中三个重要文件为``template.tex"、``ref.bib"和``testreport\_class",请确保这三个文件处于同一文件路径下。

        在编译文档时,\LaTeX{}编译格式请务必使用xelatex!
        
        \LaTeX{}编译格式请务必使用xelatex!
        
        \LaTeX{}编译格式请务必使用xelatex!

        在引用文献时,本模板使用cite宏包,bib文件是``ref.bib'',请将引用文献的正确bib格式写入该文件中。

        ``testreport\_class"文件是类文档文件,定义了本模板各种排版参数,同时提前引入了amsmath、graphics等常用宏包。如需引入其他宏包,可以在tex文件引言处直接添加。
    \section{实验装置及过程}
      可以使用figure环境轻松插入图片。如下示例:  
    
      电介质的介电常数$\varepsilon$,亦称“电容率”,是表征电介质材料绝缘性的一个主要指标。真空的介电常数$\varepsilon_{0}$是一个基本的物理学常量。
    
    %%插入图片。H代表hfill绝对位置。
		\begin{figure}[H]
			\centering
			\includegraphics[width=0.5\paperwidth]{pic/pingxing.png}
			\caption{\label{Fig.1}平行板电容器模型图}
		\end{figure}
    
      这里可以使用Fig.~\ref{Fig.1}用来引用图像。  %"~\ref{这里填入label名称}"最好别忘记"~",强制插入空格且不换行。引用前记得在相应标题处加\label{}
    \section{\label{Sec.3}实验结果和分析}
    \subsection{实验结果}
      还可以使用table环境插入表格,如果不喜欢table环境也可以使用array。以下是table环境示例:

    %%插入表格
    \begin{table}[!h]
    \centering
    \caption{\label{Tab.1}一些主要的实验数据}
    \begin{tabular}{p{0.2\textwidth}|p{0.2\textwidth}|p{0.2\textwidth}}
    \hline 
    参量1/单位 & 参量2/单位 & 参量3/单位 \\
    \hline 
    14.12 & 23.52 & 1.23 \\
    \hline
    \end{tabular}
    \end{table}

    这里可以使用Table~\ref{Tab.1}用来引用表格。
    \subsection{\label{Sec.4}数据处理与误差分析}
    测试下插入公式以及交叉引用公式。

    行内公式:$\displaystyle \frac{\partial v}{\partial y} =\frac{\partial u}{\partial x} =-6xy$且$\displaystyle \frac{\partial v}{\partial x} =-\frac{\partial u}{\partial y} =3\left( x^{2} -y^{2}\right)$. 

    行间公式:

    通过
    \begin{equation}\label{eq.1}
      f( z) =\pm\sqrt{2\rho }\left(\sin\frac{\phi }{2} +i\cos\frac{\phi }{2}\right) +iC_{2} =\pm\sqrt{2z} +iC_{2}.
    \end{equation}
    得到某样结果。

    引用公式测试,通过Eq.~\eqref{eq.1}得知。
    
    \section{\label{Sec.5}结论}
    这里测试引用文献。前人研究过这个问题\textsuperscript{\cite{yaakub_fourth_1999}}。前人研究过这个问题\textsuperscript{\cite{yang_friction_2010}}。前人研究过这个问题\textsuperscript{\cite{yang_friction_2010,_frenkel-kontorova_2014}}。
    
    \section{参考文献}
    \begin{spacing}{1}
      \nocite{*}    %用来显示所有的参考文献
      \bibliographystyle{unsrt}       %本条命令定义了文献引用格式,一般不用修改
      \bibliography{ref}
    \end{spacing}
\end{document}